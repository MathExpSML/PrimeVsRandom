    \documentclass{article}
    \usepackage[french]{babel}      %import lang norms
    \usepackage{inputenc}
    \usepackage[T1	]{fontenc}        %import encoding
    \usepackage{a4wide}
    \usepackage{nameref}            %Allows cross-ref to section name
    \usepackage{hyperref}           %Makes table of content clickable
    \usepackage{imakeidx}            % Allows index
    \usepackage{subfiles}           % Allows division into multiple pages
			\makeindex[columns=3]           % Create index
			\usepackage[totoc]{idxlayout}   % Add index to table of content
			\usepackage{graphicx}
			\usepackage[toc,page]{appendix}
    \graphicspath{ {../images/} }
    \usepackage{float}
    \usepackage{amssymb}
    \usepackage{amsthm}
		\usepackage{amsmath}
		%\setlength{\parindent}{0pt}			% Remove alinea
		\usepackage{subfigure} 	%allows multiple figures side by side
		\usepackage{listings}
		\usepackage{xcolor}

\definecolor{mygray}{rgb}{0.4,0.4,0.4}


\lstset{frame=tb,
language=R,
language=Python,
keywordstyle=\color{blue},
numbers=left,
commentstyle=\color{mygray},
showstringspaces=false,
alsoletter={.},
breaklines=true,
title=\lstname                   
}
    
\begin{document}



%Premiere page
\author{Sebastien Plaasch \\ Maxime Rubio}
\title{Primes vs Random Sets}
\maketitle
\newpage 
%\begin{abstract}
%    Resume de notre projet..
%\end{abstract}
\newpage
\tableofcontents

%\vspace{\fill} %Summary at the end of page?

\newpage

%Introduction
\section{Introduction}
    \label{sec:intro}
    %\addcontentsline{toc}{section}{\nameref{sec:intro}}
\theoremstyle{plain}
\newtheorem{Thm}{Théorème}

Le travail qui est présenté dans ce rapport porte sur les nombres premiers et plus particulièrement sur leur distribution. L'objectif est d'évaluer la place de la distribution des nombres premiers dans des conjectures sur ceux-ci. Nous désignerons par $\mathbb{P}$ l'ensemble des nombres premiers et par $\pi$ la fonction définie comme suit : $\pi : [0, \infty [ \rightarrow \mathbb{R}, x \mapsto \# \{p \in \mathbb{P}$ | $p \leqslant x\}$. 

Dans un premier temps, nous avons créer des ensembles aléatoires qui partagent la même distribution que l'ensemble des nombres premiers. Pour créer les ensembles et vérifier leur fiabilité à la distribution des nombres premiers, nous avons utilisé le théorème des nombres premiers : 
\begin{Thm}[Théorème des Nombres Premiers]
\label{TNP}
	Quand $x$ tend vers l'infini : 
	\[ \pi(x) \sim \frac{x}{\log(x)}  \] 
\end{Thm}
Nous allons aussi utiliser la fonction Li : $ [2, \infty [ \rightarrow \mathbb{R}, x \mapsto \int_{2}^{x} \frac{dt}{\log(t)} $ et la relation $\pi(x) \sim$ Li($x$) quand $x$ tend vers l'infini.  Par deux approches différentes, nous avons créé 400 ensembles aléatoires sur lesquels nous avons testé deux conjectures. 
\clearpage

\section{Création des ensembles aléatoires}
    \label{sec:sec1}
    \subfile{sections/analytic_approach}
    \subfile{sections/prob_approach}
    \subfile{sections/odd_prob_approach}
		\subfile{sections/review_sets}

\section{Ensembles aléatoires et conjectures}
    \subfile{sections/Chap2}
		\subfile{sections/Conjecture_2_3}
		
		
\section{Conclusion}
En conclusion de ce travail, on peut constater que l'approche probabiliste semble mieux fonctionner, ou du moins pour des ensembles plus petits. La distribution des ensembles $R$ est directement très proche de celle des nombres premiers tandis que pour l'approche analytique, il faut prendre des ensembles plus grands avant d'avoir des résultats probants. Néanmoins, les deux approches illustrent bien la place que prend la distribution des nombres premiers et tout ce qu'on peut faire avec grâce à tout ce qu'on sait sur la fonction $\pi$ et ses excellentes approximations. En se rapprochant davantage des nombres premiers en construisant des ensembles aléatoires composés de nombres impairs, on parvient à avoir d'excellents résultats pour les deux conjectures testées. 
\newpage

\bibliography{bibliographie.bib}
\bibliographystyle{plain}

\clearpage
\appendix
	\section{Appendice}
		\subfile{sections/appendix_code_seb}
		
%Index - le laisser a la fin
\newpage
\printindex

\end{document}