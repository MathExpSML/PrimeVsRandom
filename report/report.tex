\documentclass{article}
    \usepackage[french]{babel}      %import lang norms
    \usepackage[T1]{fontenc}        %import encoding
    \usepackage{a4wide}
    \usepackage{nameref}            %Allows cross-ref to section name
    \usepackage{hyperref}           %Makes table of content clickable
    \usepackage{imakeidx}            % Allows index
    \usepackage{subfiles}           % Allows division into multiple pages
    \makeindex[columns=3]           % Create index
    \usepackage[totoc]{idxlayout}   % Add index to table of content
    \usepackage{graphicx}
    \graphicspath{ {../images/} }
    \usepackage{float}
    \usepackage{amssymb}
    \usepackage{amsthm}
    
\begin{document}



%Premiere page
\author{Sebastien Plaasch \\ Maxime Rubio  \\ Lucas Villiere}
\title{Prime vs Random Sets}
\maketitle
\newpage 
\begin{abstract}
    Resume de notre projet..
\end{abstract}
\newpage
\tableofcontents

%\vspace{\fill} %Summary at the end of page?

\newpage

%Introductionz
\section*{Introduction}
    \label{sec:intro}
    \addcontentsline{toc}{section}{\nameref{sec:intro}}
\subfile{sections/introduction}

\section{Creation d'ensembles aléatoires suivant la distribution de $\pi(x)$}
    \label{sec:sec1}
    \subfile{sections/analytic_approach}
    \subfile{sections/prob_approach}
    \subfile{sections/odd_prob_approach}

\section{Ensembles aléatoires et conjonctures}
    \subfile{sections/Chap2}

\section*{Conclusion}
    \subfile{sections/conclusion}

%Index - le laisser a la fin
\newpage
\printindex

\end{document}