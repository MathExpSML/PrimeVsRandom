\documentclass[../main.text]{report}
\subsection{Code: Création d'ensembles aléatoires}

\subsubsection{Création d'ensembles probabiliste}
Le code suivant genere 100 ensembles aléatoires non-impairs. 
\lstinputlisting[language=R]{../r_code/generate_samples.R}

\clearpage
\subsubsection{Création d'ensembles probabiliste impairs}
Le code suivant est celui utilisé pour générer 100 ensembles aléatoires impairs.
\lstinputlisting[language=R]{../r_code/generate_samples2.R}

\clearpage
\subsection{Seconde conjecture} 
\subsubsection{Raport de conjecture}
Le code suivant génere un tableau de donnée pour chaque ensemble, contenant le nombre d'entiers vérifiant la conjecture, le nombre d'echec et le plus grand entier ne vérifiant pas la conjecture.
\lstinputlisting[language=Python]{../py_code/conjecture_2_3.py}

\clearpage
\subsubsection{Error mapping}
\label{code:Error_mapping}
Le code suivant est utilisé pour vérifier la conjecture pour un certain ensemble. Celui-ci ne génere pas de rapport mais s'arrête dès que l'assertion n'est pas vérifiée.
\lstinputlisting[language=Python]{../py_code/test_conj_2_3.py}
