\documentclass[../main.tex]{subfile}
    \label{sec:conclusion}
    \addcontentsline{toc}{section}{\nameref{sec:conclusion}}
    
En conclusion de ce travail, on peut constater que l'approche probabiliste semble mieux fonctionner, ou du moins plus tôt. La distribution des ensembles $R$ est directement très proche de celle des nombres premiers tandis que pour l'approche analytique, il faut prendre des ensembles plus grands avant d'avoir des résultats probants. Néanmoins, les deux approches illustrent bien la place que prend la distribution des nombres premiers et tout ce qu'on peut faire avec grâce à tout ce qu'on sait sur la fonction $\pi$ et ses excellentes approximations. En ce rapprochant davantage des nombres premiers en construisant des ensembles aléatoires composés de nombres impairs, on parvient à avoir d'excellents résultats pour les deux conjectures testées. 


