\documentclass[../main.text]{report}
\begin{document}
\subsection{Seconde conjecture}
\subsubsection{Introduction}

La conjoncture que nous allons tester ici est la suivante: 

	\begin{Conj}
	Pour tout $n = 6, 7, ...$, il existe un nombre premier $p$ tel que $6n-p$ et $6n+p$ sont tous les deux premiers. \footnote{Conjecture 2.3 de ...}
	\end{Conj}
	

Nous allons d'abord vérifier que cette conjecture tient pour les nombres premiers, puis vérifier si celle-ci tient aussi pour les ensembles aléatoires suivant leur distribution.
La procédure pour analyser cette conjecture est donc la suivante: pour chaque $n\in \mathbb{N}, 6 \leq n \leq 10000$, nous vérifierons si l'assertion tient. Nous collecterons alors tout $n$ tel que $\neg P(n)$ dans un tableau de données afin d'analyser, pour chaque ensemble ou chaque groupe d'ensemble, le nombre et la distribution des erreurs.

\subsubsection{Analyse}


%Utilisons les notations $R_{k_n}$, $U_{k_n}$ pour désigner les ensembles $\{i \in R_k ~|~i < n \}$ (respectivement, $\{i \in R_k ~|~i < n \}$).
Soit $P(n)$ l'assertion "Il existe $p \in R_{k_n}, p \leq n$ tel que $6n-p \in R_k$ et $6n+p \in R_k$".

En utilisant un algorithme \footnote{voir appendice: \ref{code:Error_mapping} - "py\_code/test\_conj\_2\_3.py" }, nous avons pu observer que l'assertion est vraie pour tout $n \leq 10^6$.
Le même algorithme confirme que la conjecture n'est pas vérifiée, du moins pour tout $n \geq 6$, en ce qui concerne les ensembles aléatoires. Cependant, il semble que certains de ces ensembles possèdent des propriétés similaires si l'on choisit un $n$ plus grand.

Nous nous intéressons au nombre d'erreurs (c'est à dire le nombre d'entiers $n$ pour lesquelle l'assertion n'est pas vérifiée), ainsi que le plus grand entier pour lequel l'assertion est fausse. Cette dernière information est intéressante car si ce plus grand entier est petit, alors la conjoncture est vérifiée pour tout $n$ plus grand.

Le graphique \ref{fig:failures_2_3} ci-dessous représente en abscisses le nombre total d'erreur pour $n \in \{6,...,10^5\}$, et en ordonées le plus grand entier pour laquelle l'assertion est fausse. Chaque point représente un ensemble. Les ensembles ayant les meilleurs "performances" sont alors situés en bas à gauche: ceux-ci ont alors un faible nombre d'erreurs, et vérifie l'assertion pour tout $n$ plus grand. 

\begin{figure}[H]
\centering
\includegraphics{plot_failure_maxfaile_c_2_3}
\caption{}
\label{fig:failures_2_3}
\end{figure}

Nous debuterons par remarquer que l'existence d'un entier $p$ répondant aux critères de la conjecture n'est pas rare. En effet, pour $ n \in \{6,...,10000\}$, l'assertion est vérifiée par près de 99\% des entiers $n$ testés pour chaque ensemble (moins de 650 erreurs). 

Les ensembles non restreints aux nombres impairs génerent tout de même un nombre assez élevé d'erreurs: plus de 200 erreurs pour la quasi-totalité de ces ensembles.
De plus, ces erreurs persistent assez tardivement: pour la majorité de ces ensembles, il existe (au moins) un entier $n > 5000$ pour lequel l'assertion est fausse. 


Les ensembles composés de nombres impairs ont cependant de bien meilleurs performances . Ceux-ci ont un faible nombre d'erreurs (moins de 50 pour les ensembles générés par l'algorithme probabilistique, moins de 100 pour les ensembles générés par les algorithmes analytiques, voir figure \ref{fig:boxplots_Odd}). 
De plus, le plus grand entier pour lequel l'assertion n'est pas vérifiée est relativement faible: cela signifie que pour tout entier $n > 1500$, l'assertion est vérifiée. Pour plus de trois quarts des ensembles aléatoires générés par l'algorithme probabiliste, on a même l'assertion vérifiée pour tout $n > 500$.
Finalement, on remarque aussi que les ensembles crées par l'algorithme probabiliste ont des performances sensiblement meilleures. Cela est surement du au fait qu'ils possèdent sensiblement plus d'élements que les autre ensembles (voir figure \ref{fig:comparison_all_random_sets}), ou alors parce que leur distribution est plus proche de celles des nombres premiers.

Les diagrammes à boites \ref{fig:boxplots} et \ref{fig:boxplots_Odd} ci-dessus offrent un aperçu de la distribution des erreurs. La seconde figure se concentre sur les ensembles impairs afin de faciliter la lecture du graphique.

\begin{figure}[H]
\centering
	\subfigure[Plus grand n ne vérifiant pas l'assertion]{\includegraphics{C23_MaxFail_BoxPlot}}%
	\subfigure[Nombre d'erreurs]{\includegraphics{C23_FailCount_BoxPlot}}
	\caption{Diagrammes en boite}
	\label{fig:boxplots}
\end{figure}

\begin{figure}[H]
\centering
	\subfigure[Plus grand n ne vérifiant pas l'assertion]{\includegraphics{C23_MaxFail_BoxPlot_Odd}}%
	\subfigure[Nombre d'erreurs]{\includegraphics{C23_FailCount_BoxPlot_Odd}}
	\caption{Diagrammes en boite - ensembles impairs.}
	\label{fig:boxplots_Odd}
\end{figure}

Il existe trois ensembles vérifiant l'assertion pour tout $n, 100 \leq n \leq 10000$:
\begin{itemize}
	\item l'ensemble probabiliste impair 032 vérifiant l'assertion pour tout $n > 38$
	\item l'ensemble probabiliste impair 091 vérifiant l'assertion pour tout $n > 74$
	\item l'ensemble probabiliste impair 033 vérifiant l'assertion pour tout $n > 97$
\end{itemize}
Pour ces ensembles, l'assertion tient au moins jusqu'à $n=10^6$.  

De plus, au delà de 100, il existe 50 ensembles ayant moins de 5 erreurs. 
Il apparait alors que la probabilité d'une erreur diminue, et tend vers 0, lorsque n devient grand, comme le font remarquer les graphiques ci-dessous
\begin{figure}[H]
\centering
\includegraphics{conjecture_2_3_10k_bin5_line}
\caption{Diagramme représentant le nombre moyen d'erreur pour chaque entier $1 \leq n < 10000$}
\label{fig:conjecture_2_3_10k_bin5_line}
\end{figure}

On observe ici que pour les ensembles aléatoires impairs, le nombre moyen d'erreur tends très vite vers 0 jusqu'à qu'aucune erreur n'apparaisse aux alentours de 1200. Cette statistique diminue moins vite pour les autres ensemble, mais tend aussi vers 0.
En effet, il apparait que pour un $n$ suffisement grand, la probabilité que l'assertion ne soit pas vérifiée est négligeable. 

\subsubsection{Conclusion de l'analyse}
Il semble que les ensembles aléat

possible que pour les nombres premiers, comme pour l'ensemble probabiliste impair 032, que la conjecture ne soit vraie qu'à cause d'une distribution "heureuse" au départ, lorsque $n$ est suffisement petit, et que pour tout $n$ suffisement grand, la probabilité d'une erreur est marginale. 


%\begin{figure}[h]
%\centering
%\includegraphics{heat_conjecture_2_3}
%\caption{diagramme représentant le nombre moyen d'erreur (entier ne vérifiant pas la conjecture) sur l'interval $(n,n+5]$}
%\label{fig:heat_conjecture_2_3		}
%\end{figure}

%Cela semble avoir du sens dans la mesure où le nombres d'elements $p$ continue d'augmenter lorsque p augmente, 



\end{document}
