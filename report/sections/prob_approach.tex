\documentclass[../main.tex]{report}

\begin{document}
    \label{sec:test1}
    
\subsection{Approche probabiliste}
Soient $E_n := \left\{ k \in N^* ~|~k < n \right\}$ l'ensemble des entiers inférieurs à n, $ P_n := \{k \in E_n~|~ k$ est premier$ \}$ l'ensemble des nombres premiers inférieurs à n, et la fonction $\pi (n) := \#P_n$, le nombres de premiers inferieurs à n.

On a vu que la fonction Li$(x) = \int_2^x \frac{dt}{\log t}$ 
donne une bonne approximation de $\pi(n)$.

Cette fonction peut être approximée par la somme de Riemann de pas contant $=1$:
\begin{equation}
\label{eq:SommeDeRiemann}
S \left({\frac{1}{\log x}} \right)
= \sum_{k=0}^{n-1} \frac{1}{\log(2 + k)}
= \frac{1}{\log 2} + \sum_{k=3}^{n-1} \frac{1}{\log k}
\end{equation}


La fonction $\frac{1}{\log x}$ est une fonction continue, décroissante et positive sur l'interval $\left[2, \infty \right[$. 
L'erreur entre Li$(n)$ et la fonction en escalier ci-dessus est donc bornée. 
 \[ 
\left| S(\frac{1}{\log x}) - Li(n) \right|
\leq \left| \sum_{k=2}^{n-1} \frac{1}{\log k} - \frac{1}{\log (k+1)} \right| 
 %= \left| \frac{1}{\log (n+1)} - \frac{1}{\log 2} \right|
 = \frac{1}{\log (2)} - \frac{1}{\log (n)}
 < \frac{1}{\log 2}
 < 2
 \]


\subsubsection{Ensembles aléatoires} 

Nous allons alors générer des ensembles aléatoires $R_{k} \subset \mathbb{N}, k \in \{1,...,100\}$ de sorte que:
\[
\forall~i \in \mathbb{N}, P(i \in R_{k}) = 
\left\{ 
    \begin{array}{cl}
         0 & \mbox{si}~i = 1 \\
         1 & \mbox{si}~i = 2 \\
         \frac{1}{\log n} & \mbox{si}~i \geq 3
    \end{array}
\right.
\]
Ces ensembles seront construits jusqu'à $i = 10^7$.

Il est à noter deux cas particuliers:
\begin{itemize} 
    \item Le nombre 1 est exclu. En effet, $\frac{1}{\log 1}$ n'est pas défini. Par définition, $1$ n'est pas un nombre premier.
    \item Le nombre 2 est inclu par défaut. En effet, $P(2 \in R_{k_n}) = \frac{1}{\log 2} > 1$. De plus, le nombre 2 est par définition, un nombre premier.
\end{itemize}


La fonction $\sigma_{R_k}(n) := \# \{i \in R_{k} | i < n\}$ mesure donc la taille des ensembles jusqu'à un certain $n$. Cette fonction mesure la taille de l'ensemble aléatoire. 
Cette fonction est donc une valeur aléatoire strictement inférieure à $n$ dont l'espérance est donnée par la formule suivante:
\begin{equation}
\label{eq:esperance}
E[\sigma_{R_k}(n)] = 0 + 1 + \frac{1}{\log 3} + ... + \frac{1}{\log (n-1)}
= 1 + \sum_{k=3}^{n-1} \frac{1}{\log k}
\end{equation}

Notons $\sigma_R$ (sans indicer $R$) l'ésperance de cette fonction.
On observe alors que l'erreur entre $\sigma_R(n)$ et la somme de Riemann (\ref{eq:SommeDeRiemann}) est constante, égale à
$\frac{1}{\log 2}- 1 < 1$.

Les ensembles aléatoires générés de cette manière suivront donc une distribution similaire à Li$(x)$, et donc à $\pi(x)$
(voir figures \ref{fig:comparison_sigma_prob} et \ref{fig:prob_sample}).

\begin{figure}[H]
\includegraphics[width=\textwidth]{comparison_sigma_prob}

\caption{Graphes des fonctions $\pi$, Li and $\sigma$. Li et $\sigma$ sont superposées.}
\label{fig:comparison_sigma_prob}
\end{figure}

\begin{figure}[H]
	\centering
	\includegraphics[width=\textwidth]{prob_samples}
	\caption{graphes des fonctions $\sigma_{R_k}$ pour $k \leq 25$ (i.e. les 25 premiers ensembles) et $\pi$ (en pointillé)}
	\label{fig:prob_sample}
\end{figure}
\end{document}
